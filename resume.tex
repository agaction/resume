%%%%%%%%%%%%%%%%%%%%%%%%%%%%%%%%%%%%%%%%%
% Medium Length Professional CV
% LaTeX Template
% Version 2.0 (8/5/13)
%
% This template has been downloaded from:
% http://www.LaTeXTemplates.com
%
% Original author:
% Rishi Shah 
%%%%%%%%%%%%%%%%%%%%%%%%%%%%%%%%%%%%%%%%%

% PACKAGES AND OTHER DOCUMENT CONFIGURATIONS

\documentclass{template} % Use the custom resume.cls style

\usepackage[left=0.5in,top=0.5in,right=0.5in,bottom=0.5in]{geometry}
\usepackage{fontawesome} % Document margins
\usepackage{hyperref}
\newcommand{\tab}[1]{\hspace{.2667\textwidth}\rlap{#1}}
\newcommand{\itab}[1]{\hspace{0em}\rlap{#1}}
\name{Akhil Goel}
%\address{7700 Falstaff Rd, McLean, VA 22102}
\address{\href{mailto:akhil.goel.2000@gmail.com}{akhil.goel.2000@gmail.com } \\ \space \faPhone \space {(571)\:353-9527} \\ \space \faGithub \space \href{https://agaction.github.io}{agaction }}

\begin{document}

% EDUCATION SECTION

\begin{rSection}{Education}
{\bf Georgia Institute of Technology, Atlanta, GA} \hfill {\em Aug 2018 - July 2023}
\\ {\em M.S. in Computer Science - Machine Learning} \hfill  GPA: 3.5/4.0
\vspace{0.1cm}
\\ \underline{Notable Coursework:} Machine Learning (ML), Natural Language Processing (NLP), Interactive Robot Learning (Reinforcement Learning), Convex Optimization, High Performance Computing (HPC), Quantum Computing, Dynamic Algebraic Algorithms, Cryptography \& Blockchain, ML for Trading/Computational Biology 

{\em B.S. in Mathematics \& Computer Science} \hfill  GPA: 3.67/4.0
\vspace{0.1cm}
\\ \underline{Notable Coursework:} Data Structures, Advanced Algorithms, Object-oriented Programming (OOP), Deep Learning, Computer Vision, Robotics, Artificial Intelligence (AI), Computer Architecture, Systems, Databases, Computer Networking, Information Security, Automata and Complexity, Linear Algebra, Vector Calculus, Probability, Statistics, Stochastic Processes, Algebra, Real/Complex/Combinatorial/Numerical Analysis
\vspace{0.20cm}
\\ {\bf Thomas Jefferson High School for Science and Technology (TJHSST)} \hfill {\em Aug 2014 - May 2018}
\end{rSection}

% WORK EXPERIENCE

\begin{rSection}{Work Experience}

\begin{rSubsection}{Google LLC}{May 2022 - August 2022}{Software Engineering Intern}{Mountain View, CA}
\item Designed and implemented subscription free-trial abuse detection algorithms on Google Play Store.
\item Leveraged asynchronous gRPC protocols to enable high-throughput abuse checks with little to no latency cost
\item Analyzed revenue impact of abuse checks and created user experiments to enable ramp-up of abuse verdicts.
\end{rSubsection}

\begin{rSubsection}{Meta (Facebook AI Applied Research)}{May 2021 - August 2021}{Software Engineering Intern}{Menlo Park, CA}
\item Developed influential instance model interpretability algorithms for \href{https://captum.ai/}{Captum,} an open-source PyTorch library.
\item Implemented novel algorithms from recent \href{https://arxiv.org/pdf/2002.08484.pdf}{research publications} and created \href{https://captum.ai/tutorials/TracInCP_Tutorial}{tutorials} for influence methods.
\item Designed rigorous testing frameworks and applied algorithms to large Facebook AI Multimodal models.
\end{rSubsection}

\begin{rSubsection}{Varian Medical Systems}{May 2020 - August 2020}{Software Engineering Intern}{Atlanta, GA}
\item Built automized data preprocessing pipeline for developing local deep learning segmentation models.
\item Developed deep volumetric models for organ segmentation within Varian's oncology PACS, Velocity.
\end{rSubsection}

\begin{rSubsection}{MIT Lincoln Laboratory}{June 2019 - August 2020}{Summer Research Intern \& Student Technical Researcher}{Lexington, MA}
\item Used Monte-Carlo simulations to develop a time-gating algorithm for a novel brain imaging technique.
\item Co-Authored \href{https://bit.ly/AGpaper1}{conference publication} at SPIE Photonics West, San Francisco, February 2020.
\item$1^{\text{st}}$ Place Team prize for proposal on automatic American Sign Language translation. Received funding.
\end{rSubsection}

\end{rSection}
\vspace{-2mm}

%Projects

% \begin{rSection}{Research}
% {\bf Visually Grounded PointGoal Navigation at CVMLP Lab} \hfill {\em August 2021 - May 2022} \smallskip
% \\Research on visually-grounded navigation in dynamic environments. Advised by Prof. Dhruv Batra. \vspace{3mm} \\
% {\bf Autonomous Robot Navigation at Georgia Tech IVALab} \hfill {\em Jan 2019 - May 2020} \smallskip
% \\Applying reinforcement learning algorithms to improve traditional local planning methods in robotics.
% \end{rSection}

%TECHNICAL STRENGTHS

\begin{rSection}{Technical Skills}

\begin{tabular}{ @{} >{\bfseries}l @{\hspace{6ex}} l }
Programming Languages & \hspace{-0.75cm} Python, C/C++, Java, CUDA, Bash, SQL, Solidity \\[2pt]
Technologies \& Tools & \hspace{-0.75cm} PyTorch, Keras, Numpy, Pandas, Git, Docker, AWS, MPI, Qiskit, HTML/CSS \\[2pt]
\end{tabular}

\end{rSection}

% Leadership & Extra Curricular

\begin{rSection}{Leadership \& Extracurricular} \itemsep -3pt \vspace{-3mm}
\item {\bf External Operations - The Agency:} President for AI/ML research club at Georgia Tech.
\item {\bf GTA - Computer Vision:} Led keypoint matching project (SIFT), taught students in office hours. 
\end{rSection}

% Awards & Achievements

% \begin{rSection}{Awards \& Achievements} \itemsep -3pt \vspace{-3mm}
% \item {\bf Qualifier} for the 2018 and 2015 American Invitational Mathematics Examination (AIME)
% \item {\bf Semifinalist (Top 500 in the nation)} for the 2017 and 2016 USA Biology Olympiad (USABO)
% \end{rSection}

\end{document}

